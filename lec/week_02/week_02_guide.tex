\documentclass{article}

%----------------------------------------------------------------------------------------
\usepackage{enumerate}
\usepackage{listings} % Required for inserting code snippets
\usepackage[usenames,dvipsnames]{color} % Required for specifying custom colors and referring to colors by name

\definecolor{DarkGreen}{rgb}{0.0,0.4,0.0} % Comment color
\definecolor{highlight}{RGB}{255,251,204} % Code highlight color

\lstdefinestyle{Style1}{ % Define a style for your code snippet
language=Python, % Detects keywords, comments, strings, functions, etc for the language specified
backgroundcolor=\color{highlight}, % Set the background color for the snippet - useful for highlighting
basicstyle=\footnotesize\ttfamily, % The default font size and style of the code
breakatwhitespace=false, % If true, only allows line breaks at white space
breaklines=true, % Automatic line breaking (prevents code from protruding outside the box)
captionpos=b, % Sets the caption position: b for bottom; t for top
commentstyle=\usefont{T1}{pcr}{m}{sl}\color{DarkGreen}, % Style of comments within the code - dark green courier font
deletekeywords={}, % If you want to delete any keywords from the current language separate them by commas
%escapeinside={\%}, % This allows you to escape to LaTeX using the character in the bracket
firstnumber=1, % Line numbers begin at line 1
frame=single, % Frame around the code box, value can be: none, leftline, topline, bottomline, lines, single, shadowbox
frameround=tttt, % Rounds the corners of the frame for the top left, top right, bottom left and bottom right positions
keywordstyle=\color{Blue}\bf, % Functions are bold and blue
morekeywords={}, % Add any functions no included by default here separated by commas
numbers=left, % Location of line numbers, can take the values of: none, left, right
numbersep=10pt, % Distance of line numbers from the code box
numberstyle=\tiny\color{Gray}, % Style used for line numbers
rulecolor=\color{black}, % Frame border color
showstringspaces=false, % Don't put marks in string spaces
showtabs=false, % Display tabs in the code as lines
stepnumber=5, % The step distance between line numbers, i.e. how often will lines be numbered
stringstyle=\color{Purple}, % Strings are purple
tabsize=4, % Number of spaces per tab in the code
}

% set title
\title{DSC 232R: Big Data Analytics Using Spark \\ Winter 2026 \\ Week 2} 
%----------------------------------------------------------------------------------------

\begin{document}

\maketitle

%----------------------------------------------------------------------------------------
% section 1 
%----------------------------------------------------------------------------------------

\section{Topic: HDFS and The MapReduce Paradigm}

%----------------------------------------------------------------------------------------
% section 1.1
%----------------------------------------------------------------------------------------
\subsection{The Distributed File System (HDFS)}
\subsubsection{Lecture Content}
\begin{itemize}
    \item \textbf{The Core Problem:} Standard file systems cannot handle petabytes of data because they rely on a single disk/computer.
    \item \textbf{Google's Solution (GFS/HDFS):}
    \begin{enumerate}
        \item \textbf{Commodity Hardware:} Use thousands of cheap, unreliable computers instead of one expensive supercomputer.
        \item \textbf{Chunking:} Files are broken into fixed-size blocks (default 128MB) called "Chunks" or "Splits".
        \item \textbf{Redundancy (Replication):} Every chunk is copied to multiple machines (Default Replication Factor = 3).
        \begin{itemize}
            \item \textit{Why?} If a machine crashes (which happens constantly), the data exists elsewhere.
        \end{itemize}
        \item \textbf{Architecture:}
        \begin{itemize}
            \item \textbf{NameNode (Master):} Stores metadata (filename $\to$ list of chunk IDs).
            \item \textbf{DataNode (Worker):} Stores the actual raw bytes.
        \end{itemize}
    \end{enumerate}
    \item \textbf{Locality Principle:} "Move computation to the data, not data to the computation."
    \begin{itemize}
        \item Network bandwidth is the bottleneck. It is faster to send a 5KB program to the data than to pull 1TB of data to the program.
    \end{itemize}
\end{itemize}

%----------------------------------------------------------------------------------------
% section 2
%----------------------------------------------------------------------------------------
\section{Topic: Spark Architecture}

%----------------------------------------------------------------------------------------
% section 2.1 
%----------------------------------------------------------------------------------------
\subsection{RDDs and Context}
\subsubsection{Lecture Content}
\begin{itemize}
    \item \textbf{SparkContext (The Driver):}
    \begin{itemize}
        \item The "Brain" of the operation. Resides on the Master node.
        \item Does NOT store the data. It stores the \textit{Execution Plan} (Lineage).
        \item Coordinates workers (Executors).
    \end{itemize}
    \item \textbf{RDD (Resilient Distributed Dataset):}
    \begin{itemize}
        \item \textbf{Resilient:} Reconstructs lost data automatically using lineage (re-computing steps) rather than replication.
        \item \textbf{Distributed:} Partitioned across the cluster.
        \item \textbf{Immutable:} Once created, it cannot be changed. You can only create a \textit{new} RDD from an old one.
    \end{itemize}
    \item \textbf{Spark vs. Hadoop:}
    \begin{itemize}
        \item \textbf{Hadoop MapReduce:} writes to Disk after every step (High Latency).
        \item \textbf{Spark:} keeps data in Memory (RAM) between steps (Low Latency).
    \end{itemize}
\end{itemize}

%----------------------------------------------------------------------------------------
% section 2.2
%----------------------------------------------------------------------------------------
\subsection{Lazy Evaluation and Pipelines}
\subsubsection{Lecture Content}
\begin{itemize}
    \item \textbf{Two Types of Operations:}
    \begin{enumerate}
        \item \textbf{Transformations (Lazy):} Definition of a plan. Returns a new RDD.
        \begin{itemize}
            \item \textit{Examples:} \texttt{map}, \texttt{filter}, \texttt{flatMap}, \texttt{sample}.
            \item \textit{Behavior:} Spark does NOTHING when these lines are run. It builds a DAG (Directed Acyclic Graph).
        \end{itemize}
        \item \textbf{Actions (Eager):} Triggers execution. Returns a result to the Driver.
        \begin{itemize}
            \item \textit{Examples:} \texttt{collect}, \texttt{count}, \texttt{reduce}, \texttt{take}, \texttt{saveAsTextFile}.
            \item \textit{Behavior:} Spark optimizes the DAG and launches tasks on the cluster.
        \end{itemize}
    \end{enumerate}
    \item \textbf{Pipelining (Fusion):}
    \begin{itemize}
        \item Because of lazy evaluation, Spark can combine multiple steps into one pass.
        \item \textit{Example:} \texttt{map} + \texttt{filter} happens in a single loop over the data, avoiding intermediate memory storage.
    \end{itemize}
    \item \textbf{Caching:}
    \begin{itemize}
        \item Because RDDs are re-computed by default, branching (using an RDD twice) causes double computation.
        \item \texttt{rdd.cache()} tells Spark to save the result in RAM after the first computation.
    \end{itemize}
\end{itemize}

\newpage
%----------------------------------------------------------------------------------------
% section 3
%----------------------------------------------------------------------------------------
\section{Topic: Key-Value Operations (The Danger Zone)}

%----------------------------------------------------------------------------------------
% section 3.1
%----------------------------------------------------------------------------------------
\subsection{Pair RDDs}
\subsubsection{Jupyter Notebook Content}
\begin{itemize}
    \item \textbf{Data Structure:} RDDs where elements are Python tuples: \texttt{(Key, Value)}.
    \item \textbf{reduceByKey(func):}
    \begin{itemize}
        \item Groups values by key and applies \texttt{func} (e.g., sum).
        \item \textbf{Architecture Win:} Performs \textit{Map-Side Combination}. It reduces data LOCALLY before sending it over the network.
        \item \textit{Result:} Low network traffic, high performance.
    \end{itemize}
    \item \textbf{groupByKey():}
    \begin{itemize}
        \item Groups values by key but performs NO reduction. Returns \texttt{(Key, Iterable)}.
        \item \textbf{Architecture Fail:} Sends ALL raw data over the network to the reducer.
        \item \textit{Result:} High network traffic, likely to cause \textbf{Out Of Memory (OOM)} errors on the reducer.
    \end{itemize}
    \item \textbf{The Shuffle:}
    \begin{itemize}
        \item Operations that require moving data between partitions (by key) cause a "Shuffle".
        \item This breaks the pipeline and defines a "Stage Boundary".
        \item \textit{Shuffle Ops:} \texttt{reduceByKey}, \texttt{groupByKey}, \texttt{join}, \texttt{repartition}.
    \end{itemize}
\end{itemize}

%----------------------------------------------------------------------------------------
% Exam Traps Section
%----------------------------------------------------------------------------------------
\section{Exam Traps: Danger Zone}
\begin{itemize}
    \item \textbf{The "Average" Trap (Algebraic Properties):}
    \begin{itemize}
        \item You cannot calculate an average using \texttt{reduce(lambda x,y: (x+y)/2)}.
        \item \textbf{Why?} Reduce operations must be \textbf{Associative} and \textbf{Commutative}.
        \item $(a+b)/2 + c$ is not the same as $(a+b+c)/3$.
        \item \textbf{Solution:} Map to \texttt{(sum, count)} tuples first, reduce by adding components element-wise, then divide at the very end.
    \end{itemize}
    \item \textbf{The groupByKey Trap:}
    \begin{itemize}
        \item Prof will ask: "Why did my cluster crash when I switched from \texttt{reduceByKey} to \texttt{groupByKey}?"
        \item \textbf{Answer:} \texttt{groupByKey} forces a massive shuffle of all data. \texttt{reduceByKey} pre-aggregates on the map side (like a combiner), sending minimal data.
    \end{itemize}
    \item \textbf{The Collect Trap:}
    \begin{itemize}
        \item \texttt{rdd.collect()} brings ALL data to the Driver node.
        \item If the RDD is 1TB and the Driver has 16GB RAM, the driver crashes.
        \item \textbf{Exam Fix:} Use \texttt{take(n)} or \texttt{saveAsTextFile()} instead.
    \end{itemize}
    \item \textbf{Terminology Shift:}
    \begin{itemize}
        \item Lecture: "Lazy Evaluation". Exam: "DAG Construction vs. Materialization".
        \item Lecture: "Shuffle". Exam: "Wide Dependency".
        \item Lecture: "Chunk". Exam: "Partition" or "Split".
    \end{itemize}
\end{itemize}

\newpage
%----------------------------------------------------------------------------------------
% Practice Quiz Section
%----------------------------------------------------------------------------------------
\section{Practice Quiz}

% Question 1
\subsection*{Question 1}
\subsubsection*{The Philosophy of MapReduce}
Which statement best describes the "Data Locality" principle in HDFS/Spark?
\begin{enumerate}[a]
    \item All data is gathered to the Master node for processing.
    \item Data is randomly shuffled across the network to ensure fairness.
    \item Code (computation) is sent to the node where the data resides.
    \item Data is always compressed before processing.
\end{enumerate}

\subsubsection*{Answer: C}
\subsubsection*{Brief Explanation}
\begin{itemize}
    \item Network bandwidth is the scarcest resource. Moving code (kilobytes) to the data (terabytes) is orders of magnitude faster than moving data to the code.
\end{itemize}

% Question 2
\subsection*{Question 2}
\subsubsection*{The Math of Reduce}
Why does the operation \texttt{rdd.reduce(lambda a,b: a - b)} yield unpredictable results?
\begin{enumerate}[a]
    \item Subtraction is too computationally expensive.
    \item Subtraction is not Associative.
    \item Spark does not support negative numbers.
    \item The lambda syntax is incorrect.
\end{enumerate}

\subsubsection*{Answer: B}
\subsubsection*{Brief Explanation}
\begin{itemize}
    \item Reduce operations execute in parallel in undefined orders. $(a-b)-c$ is not the same as $a-(b-c)$. Operations must be Associative and Commutative to guarantee deterministic results in a distributed system.
\end{itemize}

% Question 3
\subsection*{Question 3}
\subsubsection*{Lazy Evaluation}
You run the following code on a 10TB dataset:
\begin{lstlisting}[style=Style1]
data = sc.textFile("huge_file.txt")
mapped = data.map(lambda x: x.split())
filtered = mapped.filter(lambda x: len(x) > 5)
\end{lstlisting}
How much time does this take to execute?
\begin{enumerate}[a]
    \item Approx 1 hour (reading 10TB).
    \item Approx 10 minutes (filtering reduces size).
    \item Microseconds (almost instant).
    \item It depends on the number of executors.
\end{enumerate}

\subsubsection*{Answer: C}
\subsubsection*{Brief Explanation}
\begin{itemize}
    \item \texttt{textFile}, \texttt{map}, and \texttt{filter} are all \textbf{Transformations}. Spark is Lazy—it only builds the Execution Plan (DAG) in memory. No data is read until an Action (like \texttt{count}) is called.
\end{itemize}

% Question 4
\subsection*{Question 4}
\subsubsection*{Performance Tuning}
You need to count word frequencies. Which operation is more efficient and why?
\begin{enumerate}[a]
    \item \texttt{groupByKey().map(sum)}
    \item \texttt{reduceByKey(sum)}
\end{enumerate}

\subsubsection*{Answer: B}
\subsubsection*{Brief Explanation}
\begin{itemize}
    \item \texttt{reduceByKey} performs \textbf{Map-Side Combination}. It sums the counts locally on each worker \textit{before} sending data across the network. \texttt{groupByKey} shuffles every single word pair, causing massive network congestion.
\end{itemize}

% Question 5
\subsection*{Question 5}
\subsubsection*{Stage Boundaries}
What triggers the creation of a new "Stage" in a Spark Job?
\begin{enumerate}[a]
    \item Any Transformation.
    \item Any Action.
    \item A Shuffle (Wide Dependency).
    \item A Cache command.
\end{enumerate}

\subsubsection*{Answer: C}
\subsubsection*{Brief Explanation}
\begin{itemize}
    \item Spark pipelines "Narrow Dependencies" (map, filter) into a single stage. A "Shuffle" (reduceByKey, join, repartition) requires data to move between partitions, forcing a hard barrier (Stage Boundary) where all previous tasks must finish before the next begins.
\end{itemize}

\newpage

% Question 6
\subsection*{Question 6}
\subsubsection*{HDFS Reliability}
If a DataNode crashes, how does HDFS ensure data is not lost?
\begin{enumerate}[a]
    \item It uses RAID 5 on the Master Node.
    \item It recovers the data from the NameNode's RAM.
    \item It relies on the 3x Replication Factor (copies exist on other nodes).
    \item The job simply fails.
\end{enumerate}

\subsubsection*{Answer: C}
\subsubsection*{Brief Explanation}
\begin{itemize}
    \item HDFS replicates every chunk (default 3 times) across different nodes. If one node dies, the Master simply redirects requests to one of the replicas.
\end{itemize}

% Question 7
\subsection*{Question 7}
\subsubsection*{Driver vs. Executor}
What is the primary risk of running the following command on a production cluster?
\texttt{result = huge\_rdd.collect()}
\begin{enumerate}[a]
    \item It will delete the original data on HDFS.
    \item It forces the Executors to crash.
    \item It causes an Out Of Memory (OOM) error on the Driver.
    \item It runs too slowly because of Python serialization.
\end{enumerate}

\subsubsection*{Answer: C}
\subsubsection*{Brief Explanation}
\begin{itemize}
    \item \texttt{collect()} fetches all data from all distributed partitions and tries to fit it into the memory of the single Driver machine. If the data > Driver RAM, the driver crashes.
\end{itemize}

% Question 8
\subsection*{Question 8}
\subsubsection*{DAG Visualization}
What is the purpose of \texttt{rdd.toDebugString()}?
\begin{enumerate}[a]
    \item To print the first 10 rows of data.
    \item To view the logical Execution Plan (Lineage).
    \item To check for syntax errors in Python.
    \item To check the size of the RDD in bytes.
\end{enumerate}

\subsubsection*{Answer: B}
\subsubsection*{Brief Explanation}
\begin{itemize}
    \item \texttt{toDebugString()} outputs the DAG (Directed Acyclic Graph) showing the lineage of transformations and where shuffles/caching will occur.
\end{itemize}

% Question 9
\subsection*{Question 9}
\subsubsection*{Glom}
What does the \texttt{glom()} transformation do?
\begin{enumerate}[a]
    \item Deletes empty partitions.
    \item Coalesces all partitions into one.
    \item Transforms each partition into a list (array) of elements.
    \item Sorts the data globally.
\end{enumerate}

\subsubsection*{Answer: C}
\subsubsection*{Brief Explanation}
\begin{itemize}
    \item \texttt{glom()} maps each partition to a single array containing all elements of that partition. It is useful for efficient batch operations or checking partition skew (e.g., \texttt{glom().map(len).collect()}).
\end{itemize}

% Question 10
\subsection*{Question 10}
\subsubsection*{Immutability}
You want to add 5 to every element in an RDD `A`. You run `A.map(lambda x: x+5)`. What happens to `A`?
\begin{enumerate}[a]
    \item `A` is modified in place.
    \item `A` is deleted and replaced by the new result.
    \item `A` remains unchanged; a new RDD is returned.
    \item It depends on whether `A` is cached.
\end{enumerate}

\subsubsection*{Answer: C}
\subsubsection*{Brief Explanation}
\begin{itemize}
    \item RDDs are \textbf{Immutable}. You cannot change an RDD. Transformations always return a \textit{pointer} to a new RDD that depends on the old one.
\end{itemize}

\end{document}
